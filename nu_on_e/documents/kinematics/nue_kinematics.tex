\documentclass{beamer}
\beamertemplatenavigationsymbolsempty
\usefonttheme[stillsansseriflarge]{serif}
\setbeamertemplate{footline}[frame number]
\usepackage[type1]{libertine}
\usepackage{bbold}
\usepackage{tikz}
%\usetheme{Warsaw}
% \usetheme{Goettingen} alternative themes
% \usetheme{Hannover}
% \usetheme{Marburg}
% \usetheme{PaloAlto}
%\title[Geek’s Night : Velha-A-Branca – March 2, 2010]{Introduction of Aspect Oriented Programming}
\author{Shih-Kai Lin}
%\institute[UM]{University of Minho}
\date{}
\title{\texorpdfstring{$\nu e$}{nu-e} kinematics}
\begin{document}
\begin{frame}
\titlepage
\end{frame}
\begin{frame}[allowframebreaks]{}

\begin{figure}
\centering
	\begin{tikzpicture}
	  \draw [>=latex, <->] (-1,0) node [right] {$z$} -- (-1.5,0) -- (-1.5,.5) node [above] {$x$};
	  \draw [>=latex, ->, very thick, red] (.9,1) node [left] {$(E_\nu,\vec{p}_\nu)$} -- (2.9,1);
	  \draw [dashed] (3,1) -- +(3,0);
	  \draw [>=latex, ->, very thick, red] (3,1) -- +(1.5,1.5) node [above right] {$(E',\vec{p}'_\nu)$};
	  \draw (3.3,1) arc (0:45:0.3) node [right] {$\phi$};
	  \draw [>=latex, ->, very thick, cyan] (3,1) -- +(2.598,-1.5) node [below right] {$(E_e,\vec{p})$};
	  \draw (3.6,1) arc (0:-30:.6) node [midway, right] {$\theta$};
  \end{tikzpicture}
\end{figure}


A neutrino collides with an electron at rest. Write down the 4-momenta of the neutrino and the electron in the lab frame. Suppose the neutrino is massless. Before collision we have ($\hbar=c=1$)

\framebreak

\begin{eqnarray}
\mathbb{p}_\nu &=&(E_\nu,0,0,E_\nu) \label{eq:pnu}\\
\mathbb{p}_e&=&(m,0,0,0) \label{eq:pe}
\end{eqnarray}
, where $m$ is the electron rest mass, and $E_\nu$ is the total energy of the incident neutrino.
After collision, we have
\begin{eqnarray}
\mathbb{p'}_\nu &=&(E',E'\sin\phi,0,E'\cos\phi) \label{eq:ppnu} \\
\mathbb{p'}_e &=& (E_e,-p\sin\theta,0,p\cos\theta) \label{eq:ppe}
\end{eqnarray}


The total 4-momentum before collision is
\begin{equation} \label{eq:ptot}
  \mathbb{P} = (E_\nu+m,0,0,E_\nu)
\end{equation}
The total 4-momentum after collision is
\begin{equation} \label{eq:pptot}
  \mathbb{P}' = (E'+E_e,-p\sin\theta+E'\sin\phi,0,p\cos\theta+E'\cos\phi)
\end{equation}

\framebreak

Since total 4-momentum is conserved before and after collision, we have equations
\begin{eqnarray}
  E'+E_e &=& E_\nu+m \label{eq:E} \\
  -p\sin\theta+E'\sin\phi &=& 0 \label{eq:ptotx} \\
  p\cos\theta+E'\cos\phi &=& E_\nu \label{eq:ptotz}
\end{eqnarray}

Rearranging equations~\eqref{eq:ptotx} and~\eqref{eq:ptotz}, squaring, and adding them, we eliminate $\phi$ and obtain

\begin{equation}
  E'^2 = E^2_\nu -2E_\nu p\cos\theta+p^2
\end{equation}
If we eliminate $E'$ with Eq.~\eqref{eq:E}, and employ the energy-momentum relation $E_e^2=p^2+m^2$ once, we obtain
\begin{equation} \label{eq:lasteq}
  p\cos\theta=(E_e-m)\left( 1+\frac{m}{E_\nu} \right)
\end{equation}

\framebreak

By squaring Eq.~\eqref{eq:lasteq}, and employing the energy-momentum relation again, we arrive at the answer
\begin{equation}
  \frac{E_e}{m}=\frac{\left( 1+\frac{m}{E_\nu} \right)^2+\cos^2\theta}{\left( 1+\frac{m}{E_\nu} \right)^2-\cos^2\theta}
\end{equation}

\end{frame}
\end{document}